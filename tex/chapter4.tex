%# -*- coding: utf-8-unix -*-
%%==================================================
\chapter{动态shape}

随着在深度学习领域,在动态推理场景的业务需求的快速增长,入语音视频、AI翻译等。动态shape,
成为保证AI编译技术在实际业务产品线中落地的,一大重要功能特性。

常见的深度学习编译器包括主流深度学习框架tensorflow的XLA,MindSpore的MindAKG,以及TVM技术栈都是基于静态shape语义开发的编译优化。static shape对于编译器而言,一方面可以给出更全面的优化策略以及,更极致的优化kernel性能;
另一方面在runtime时可以给出更合理的kernel调度执行策略和内存优化策略。同时static shape的缺点也很明显,由于在dynamic shape的业务场景下在编译时无法知道全部shape信息,需要对每个shape下的kernel都进行编译,这样带来的开销包括:

\begin{enumerate}
	\item 编译开销的增加。对于训练业务,编译开销导致训练迭代速度不稳定,会在训练的整个过程中产生负优化;对于推理业务,很多业务实际部署和迭代时不允许出现性能抖动,而离线的预编译又会使得部署的过程变复杂;
    \item 内存占用的增加。除了编译开销的问题之外,当shape变化范围特别大的时候,编译缓存额外的占用了内存显存,经常导致实际部署环境下的内存和显存Out of memory,严重阻碍业务的实际落地;
部分业务场景,shape变化范围可能非常大甚至是趋于无穷,例如广告推荐类业务中常见的稀疏化模型,还有分布式训练下的embedding切片等,这种情况下编译缓存永远无法收敛,用户更难以通过compiler来获取性能收益类。
\end{enumerate}

深度学习业务中的动态特性主要来源于业务中的以下几个方面:

\begin{enumerate}
	\item CV类业务:图片大小,以及数据批次都会发生变化;
    \item NLP类业务:语句的输入长度和输出长度,以及数据批次的变化;
    \item 模型计算过程中shape发生变化的场景:unique算子会导致在计算过程中shape发生变化。
\end{enumerate}