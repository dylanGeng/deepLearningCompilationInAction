%# -*- coding: utf-8-unix -*-
%%==================================================
\chapter{GPU的体系架构及CUDA编程模型}

随着人工智能特别是以GPT为代表的Generative AI的迅猛发展,GPU已经成为了一种不可或缺的工具,甚至企业都以拥有多少高端GPU作为抓住风口能力的衡量标准。
相比之下,CPU虽然在传统计算领域占据主导地位,但在处理深度学习任务时却不及GPU出色,本文将讨论:

\begin{enumerate}
	\item 为什么AI任务通常选择GPU而不是CPU;
    \item GPU在AI任务中的优势;
    \item 从GPU的体系架构上探讨从Volta到最新Hopper四代架构的演进,以及其在性能和功能上的提升。
\end{enumerate}

\section{数据通信}

\subsection{NVLink}

NVLink是什么的?以及为什么在大模型时代大家都需要用到它呢?

首先我们知道在大模型时代,通常深度学习网络都具有巨大的参数数量和复杂的结构,需要处理大量的数据。分布式训练将这些大模型分割成多个部分,由多个GPU或者计算节点之间并行处理,每个计算节点处理自己的数据子集。
然后通过全局通信,参数同步等方式进行梯度传播,此时GPU之间的通信带宽就变的越来越重要。

在NVLink出现之前,GPU与GPU之间的数据交互主要通过PCIe(Peripheral Component Interconnect Express)总线进行。但PCIe存在两个问题:

\begin{enumerate}
	\item PCIe总线的带宽相对有限,其中PCIe 4.0x16的最大带宽是64GB/s;
    \item PCIe总线的延迟相对较高,在GPU之间传输数据时,每次数据传输都需要通过CPU和主机内存来完成。
\end{enumerate}

这种传输路径会导致额外的延迟,并降低数据传输的效率。然而深度学习应用中需要更高的带宽和更低的延迟,PCIe自然时无法满足当下的AI训练任务。

NVLink利用高带宽、低延迟的通信通道,直接将多个GPU连接在一起,实现高速、高效的数据传输和共享。通过NVLink,GPU之间的数据交互可以直接在GPU之间进行,而无需通过CPU和主机内存。
这种直接访问内存(DMA)的方式大大减少了数据传输的复制和延迟,提高了数据共享的效率。此外,NVLink还提供了一致的内存空间,使得多个GPU能够共享同一份内存,简化了程序设计和数据管理的复杂性。

\begin{figure}[!htp]
    \centering
    \includegraphics[width=7cm]{example/pcie.png}
    \hspace{1cm}
    \includegraphics[width=7cm]{example/nvlink.png}
    \bicaption{中文题图}
    {English caption}
    \label{fig5}
\end{figure}

这里的主要思想就是GPU之间采用NVLink进行互联,实现GPU对卡间HBM的高速访问和PCIe进行互补。

参考文献的引用:
\begin{itemize}
    \item https://readpaper.feishu.cn/docx/UwT2dQsiko6u0RxoiXRcBtwfnAf。
    \item https://github.com/chenzomi12/DeepLearningSystem.
\end{itemize}