%# -*- coding: utf-8-unix -*-
%%==================================================
\chapter{深度学习编译概述}

深度学习编译是一个较为新颖的交叉领域,旨在将编译技术更好的为深度学习任务和系统服务。

近些年随着深度学习模型不断推陈出新。从早期的CV模型resnet,检测类模型yolo,到如今火遍大江南北的NLP类模型,transformer。
深度学习的任务越来越复杂,从小型的{bert\_small}到大型的{bert\_large},再到超大模型GPT-3。
适应与深度学习的硬件也在不断迭代,例如Google的TPU,英伟达的GPU,百度的昆仑芯,华为的达芬奇等。

人们对深度学习系统的要求也越来越高,工业界和学术界对于其研究热点也越来越多

这里会涉及到的概念和技术点也比较多,我们从上层到下,大致分为四个模块。

\begin{enumerate}
	\item 深度学习系统层(Tensorflow/Pytorch/MindSpore), system level;
    \item 图优化层, graph level;
    \item 算子优化层, operation/operator level;
    \item 硬件层, hardware level;
\end{enumerate}
