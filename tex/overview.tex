%# -*- coding: utf-8-unix -*-
\begin{overview}
\thispagestyle{empty}

看到赵捷老师的新书出版后,一股按不住的喜悦涌上心头,便开始总琢磨着自己开始撰写一本“深度学习编译技术”的著作,作为不久后即将出生的儿子的见面礼。

但是万事开头难,今天终于迈出了第一步,规划8-9月的时间来撰写,为了让自己在后续能不断的抽出时间来撰写,总得先给项目起一个好名字。一直很喜欢南京大学,这两位青年教授的文笔风格,所以套用了他们教材的前缀,
沉浸式《程序分析》教材
\footnote{\url{https://zhuanlan.zhihu.com/p/417187798}}
。暂定为《沉浸式深度学习编译实践》,期望能让每本读这本著作的读者都能有所收获,immerse,沉浸其中。

大致规划了整本书结构,后续再参考其他著作来调整。

\begin{enumerate}
	\item 深度学习编译概览
	\item 计算机体系结构发展回顾
	\item MLIR编译基础框架简介(是否也把TVM写上去)
    \item 多面体编译理论
    \item 深度学习图编译实践
    \item 多面体编译调度器实践(后面是否改名字)
    \item 通用芯片架构的高性能代码自动生成
    \item 专用芯片架构的高性能代码自动生成
    \item 深度学习编译领域速览
\end{enumerate}

如大家所见,模板的封面和扉页设计均高仿\footnote{李老师的书籍源码尚未公开,此为仿作。}自李文威老师《模形式初步》一书,并已得到李老师的使用许可;定理和定义环境则取材自网上流传的Elegantbook模版。我也从这一以模仿为主的学习过程中,对\LaTeX 有了更深入的了解。

本模板命名为$\mathbb{ Q }$-book,谐音自cubic一词。由于是一个菜鸟的作品,自然还有许多瑕疵,对此模板的错误和不足之处还请各位多多包涵。

\end{overview}
