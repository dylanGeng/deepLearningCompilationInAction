%# -*- coding: utf-8-unix -*-
%%==================================================
\chapter{深度学习图编译实践}

深度学习图优化过程中会优化深度学习模型的运行速度,内存使用情况,资源占用率等。这一过程包括图结构的调整、编译优化、子图融合,子图分割等,对于特定的硬件处理单元还有量化、剪枝等其他优化手段。

优化深度学习计算图的目的是降低模型的复杂度,使其在不用的硬件计算资源的限制下,实现最佳的性能。

总的来说,深度学习图优化是一个持续的过程,可以在模型的开发和部署的不同阶段进行。因此,熟练的掌握深度学习图优化技术是开发高效深度学习模型的关键。

下面是一些深度学习图优化的较好的方法:

\begin{enumerate}
	\item 剪枝:通过减少模型中不必要的节点和连接来减少模型大小。
	\item 量化:通过将浮点数参数转换为整数来减少模型的内存占用和计算时间。
	\item 合并:通过合并多个节点以减少模型中的节点数。
    \item 重设架构:通过更改模型的结构以改善模型的性能。
    \item 融合层:通过合并多个层以减少模型中的层数。
    \item 异构计算:通过使用异构计算设备(如GPU和CPU)来提高模型的推理效率。
    \item 动态图:通过使用动态图框架来灵活地优化模型的结构和计算。
\end{enumerate}

\section{TVM中的图编译优化}

\subsection{TVM简介}

TVM(Tesor Virtual Machine)是一个开源的深度学习编译器框架,旨在将深度学习模型部署到多种后端(如GPU,FPGA, ASIC等)上。

TVM软件栈由以下几个部分组成:

\begin{itemize}
	\item 前端:该部分可以生成模型的中间表示(IR),可以支持多种深度学习框架,如PyTorch,TensorFlow, MXNet等。
	\item 中间层:该部分包含了优化器和调度器,用于提高编译出的代码的性能。
	\item 后端:该部分包含了多种生成代码的后端,可以生成高性能的本地代码或OpenCL,CUDA等编程语言。
    \item 部署:该部分包含了与多种平台(如x86、ARM、FPGA等)进行交互的库,以实现代码的部署。
\end{itemize}

TVM软件栈的目的是提供一个通用的编译器框架,以使深度学习模型在多种平台上部署,并在每个平台上获得最佳性能。

由于TVM具有高度的可扩展性和可移植性,它被广泛应用于许多场景,如 移动设备、嵌入式系统,边缘计算等。

TVM的中间层优化主要包括以下几种:

\begin{itemize}
	\item 计算图优化:对模型的计算图进行各种优化,以提高模型的执行效率。
	\item 内存布局优化:对模型的内存布局进行调整,以获得更好的内存布局效率。
	\item 数据划分优化:对模型的数据划分进行调整,以提高数据在硬件上的并行度。
    \item 参数量化:对模型的参数进行量化,以减小模型的内存占用和带宽消耗。
    \item 硬件特定优化:根据 目标硬件的特性进行优化,以获得更好的性能。
\end{itemize}

这些优化可以按照特定的策略组合使用,以适配不同的场景。通过这些优化,TVM可以生成高效的代码,以获得最佳的部署性能。

\subsection{TVM中的Relay优化}

TVM的Relay是一个高效的图形编译系统,它提供了许多用优化模型的工具和技术。下面是一些常见的Relay图层的优化方法:

\begin{enumerate}
	\item 移除冗余节点:通过删除不 影响模型结果的节点来缩小模型大小。
	\item 量化操作:通过将浮点数参数转换为整数来减少模型的内存占用和计算时间。
	\item 合并层:通过合并多个层以减少模型中的层数。
    \item 提前预测:通过提前预测节点的结果来减少模型中的计算量。
    \item 内存优化:通过减少模型的内存占用来提高模型的性能。
    \item 参数剪枝:通过减少模型中不必要的参数来缩小模型大小。
\end{enumerate}

Relay还支持许多其他优化技术,例如图形重排,操作级内联和操作级合并等。Relay优化手段的选择取决于模型的特征和目标,通常需要调整和测试才能找到